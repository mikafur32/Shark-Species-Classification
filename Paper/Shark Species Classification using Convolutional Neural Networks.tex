\documentclass{article}
\usepackage[utf8]{inputenc}

\title{Shark Species Classification using Computer Vision Techniques}
\author{Mikey Sison and Max Curl}
\date{October 31, 2022}

\begin{document}

\maketitle

\section{Idea Proposal}
We plan on using CNN, ensemble, and/or unsupervised classification learning techniques to learn and predict species of shark based on their tooth morphology. Our data set will be obtained from the Environmental Science department in conjunction with Professor Jeffery Agnew. The data set includes a couple hundred labeled images of shark teeth. The scientific community currently relies on individuals using outdated and subjective morphometric measurements to classify shark teeth as well as intraspecies and positional variations. It is because of this that classification remains a difficult problem that we hope to shine a light on. 

\section{A Deeper Dive into Shark Teeth}
Shark teeth are some of the most abundant fossils found in the fossil record with sharks losing up to 40,000 teeth per year by modern estimates. It is because of this that shark lineages are some of the best documented lineages in the fossil record. These shark teeth are evolved in regards to the advantages given to the shark. For example, teeth with defined side-cusps are better adapted to catching more agile and fleshy fish. Conversely, serrations are thought to have evolved in conjunction with predation of larger marine mammals such as those belonging to the infraorder \textit{Cetacea}, commonly known as marine mammals.

\link https://universe.roboflow.com/search?q=shark%2520teeth


\end{document}
